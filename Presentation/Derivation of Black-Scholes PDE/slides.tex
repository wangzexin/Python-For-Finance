%%%%%%%%%%%%%%%%%%%%%%%%%%%%%%%%%%%%%%%%%
% Beamer Presentation
% LaTeX Template
% Version 1.0 (10/11/12)
%
% This template has been downloaded from:
% http://www.LaTeXTemplates.com
%
% License:
% CC BY-NC-SA 3.0 (http://creativecommons.org/licenses/by-nc-sa/3.0/)
%
%%%%%%%%%%%%%%%%%%%%%%%%%%%%%%%%%%%%%%%%%

%----------------------------------------------------------------------------------------
%	PACKAGES AND THEMES
%----------------------------------------------------------------------------------------

\documentclass{beamer}
\mode<presentation> {
\usetheme{Madrid}
}
\usepackage{url}
\usepackage{lmodern}
\usepackage{graphicx}
\usepackage{booktabs}

% for mathematics
\usepackage{amsmath}
\usepackage{amsthm}

%----------------------------------------------------------------------------------------
%	TITLE PAGE
%----------------------------------------------------------------------------------------

\title[Financial mathematics with Python]{UROPS Project Presentation 6} % The short title appears at the bottom of every slide, the full title is only on the title page

\author{Wang Zexin} % Your name
\institute[NUS]
{
Derivation of Black-Scholes PDE\\[3mm]
\medskip
\textit{Quantitative Finance\\
National University of Singapore\\}
}
\date{\today}

\begin{document}
%----------------------------------------------------------------------------------------
%	TITLE PAGE
%----------------------------------------------------------------------------------------
\begin{frame}
\titlepage
\end{frame}

%----------------------------------------------------------------------------------------
%	TABLE OF CONTENTS
%----------------------------------------------------------------------------------------

%------------------------------------------------
\begin{frame}
\frametitle{Today's Agenda}
\tableofcontents
\end{frame}

\section{Basics} %------------------------------------------------

\subsection{Black-Scholes World}

%------------------------------------------------
\begin{frame}
\frametitle{Black-Scholes World}
We assume that the following two SDE hold:
$$dM_{t}  = rM_{t}dt$$
$$dS_{t}  = \mu S_{t}dt + \sigma S_{t} dW_{t}$$
\end{frame}

\subsection{It$\hat{o}^{\prime}$s Lemma}

%------------------------------------------------
\begin{frame}
\frametitle{It$\hat{o}^{\prime}$s Lemma}
\begin{center}
We assume that the following equation hold:\\[4mm]
As $dS_{t} = \mu S_{t}dt + \sigma S_{t}dW_{t}$,
$$dV_{t} = (\frac{\partial V}{\partial t}+\frac{1}{2}\sigma^{2}S_{t}^{2}\frac{\partial V}{\partial S})dt + \frac{\partial V}{\partial S}dS_{t}$$
\end{center}
\end{frame}

\section{Derivation}

\subsection{Delta-hedging Argument}

%------------------------------------------------
\begin{frame}
\frametitle{Delta-hedging Argument}
\begin{center}
Our first aim is to find $\phi_{t}$\\[1mm]
for $\mathrm{\Pi}_{t} = V_{t} - \phi_{t}S_{t}$\\[3mm]
such that\\[1mm]
$d\mathrm{\Pi}_{t} = dV_{t} - \phi_{t}dS_{t}$(Self-financing)\\[1mm]
$d\mathrm{\Pi}_{t} = r\mathrm{\Pi}_{t}dt$(risk free)\\[3mm]
From the equations, we can obtain that\\[1mm]
$r\mathrm{\Pi}_{t}dt = dV_{t} - \phi_{t}dS_{t}$\\[1mm]
$r(V_{t} - \phi_{t}S_{t})dt = dV_{t} - \phi_{t}dS_{t}$\\[1mm]
$dV_{t}  = r(V_{t} - \phi_{t}S_{t})dt + \phi_{t}dS_{t}$\\[1mm]
\end{center}
\end{frame}

%------------------------------------------------
\begin{frame}
\frametitle{Compare with It$\hat{o}^{\prime}$s Lemma}
\begin{center}
$dV_{t}  = r(V_{t} - \phi_{t}S_{t})dt + \phi_{t}dS_{t}$\\[3mm]
By It$\hat{o}^{\prime}$s Lemma, $$dV_{t} = (\frac{\partial V}{\partial t}+\frac{1}{2}\sigma^{2}S_{t}^{2}\frac{\partial V}{\partial S})dt + \frac{\partial V}{\partial S}dS_{t}$$
Hence we obtain two equations:
$$\phi_{t} = \frac{\partial V}{\partial S}$$
$$r(V_{t} - \phi_{t}S_{t}) = \frac{\partial V}{\partial t}+\frac{1}{2}\sigma^{2}S_{t}^{2}\frac{\partial V}{\partial S}$$
\end{center}
\end{frame}

%------------------------------------------------
\begin{frame}
\frametitle{Last step of Delta-hedging}
\begin{center}
$$r(V_{t} - \frac{\partial V}{\partial S}S_{t}) = \frac{\partial V}{\partial t}+\frac{1}{2}\sigma^{2}S_{t}^{2}\frac{\partial V}{\partial S}$$
$$\frac{\partial V}{\partial t}+\frac{1}{2}\sigma^{2}S_{t}^{2}\frac{\partial V}{\partial S} + r\frac{\partial V}{\partial S}S_{t} - rV_{t} = 0$$
At time t, we have
$$\frac{\partial V}{\partial t}+\frac{1}{2}\sigma^{2}S^{2}\frac{\partial V}{\partial S} + r\frac{\partial V}{\partial S}S - rV = 0$$
\end{center}
\end{frame}

\subsection{Replication of portfolio}

%------------------------------------------------
\begin{frame}
\frametitle{First step of replication}
\begin{center}
Our aim is to find $a_{t}$ and $b_{t}$ such that\\[2mm]
$\mathrm{\Pi}_{t} = a_{t}S_{t} + b_{t}M_{t}$ can entirely replicate $V_{t}$\\[2mm]
And also, the self-financing condition holds:\\[2mm]
$d\mathrm{\Pi}_{t} = a_{t}dS_{t} + b_{t}dM_{t}$\\[5mm]
As $dS_{t} = \mu S_{t}dt + \sigma S_{t}dW_{t}$ and $dM_{t}  = rM_{t}dt$,
\begin{equation*}
\begin{split}
d\mathrm{\Pi}_{t} 
&= a_{t}(\mu S_{t}dt + \sigma S_{t}dW_{t}) + b_{t}(rM_{t}dt)\\
&= (a_{t}\mu S_{t} + rb_{t}M_{t})dt + (\sigma a_{t}S_{t})dW_{t}\\
\end{split}
\end{equation*}
\end{center}
\end{frame}

%------------------------------------------------
\begin{frame}
\frametitle{Bring in It$\hat{o}^{\prime}$s Lemma}
\begin{center}
By It$\hat{o}^{\prime}$s Lemma, 
\begin{equation*}
\begin{split}
dV_{t}
&= (\frac{\partial V}{\partial t}+\frac{1}{2}\sigma^{2}S_{t}^{2}\frac{\partial V}{\partial S})dt + \frac{\partial V}{\partial S}dS_{t}\\
&= (\frac{\partial V}{\partial t}+\frac{1}{2}\sigma^{2}S_{t}^{2}\frac{\partial V}{\partial S})dt + \frac{\partial V}{\partial S}(\mu S_{t}dt + \sigma S_{t}dW_{t})\\
&= (\frac{\partial V}{\partial t}+\frac{1}{2}\sigma^{2}S_{t}^{2}\frac{\partial V}{\partial S}+\frac{\partial V}{\partial S}\mu S_{t})dt + (\sigma S_{t}\frac{\partial V}{\partial S})dW_{t}\\
\end{split}
\end{equation*}
\end{center}
\end{frame}

%------------------------------------------------
\begin{frame}
\frametitle{Compare with previous equation}
\begin{center}
As $\mathrm{\Pi}_{t}$ fully replicates $V_{t}$,
$$d\mathrm{\Pi}_{t} = (a_{t}\mu S_{t} + rb_{t}M_{t})dt + (\sigma a_{t}S_{t})dW_{t} = dV_{t}$$
Also by It$\hat{o}^{\prime}$s Lemma, 
$$dV_{t} = (\frac{\partial V}{\partial t}+\frac{1}{2}\sigma^{2}S_{t}^{2}\frac{\partial V}{\partial S}+\frac{\partial V}{\partial S}\mu S_{t})dt + (\sigma S_{t}\frac{\partial V}{\partial S})dW_{t}$$
Hence we obtain that,
$$a_{t} = \frac{\partial V}{\partial S}$$
$$a_{t}\mu S_{t} + rb_{t}M_{t} = \frac{\partial V}{\partial t}+\frac{1}{2}\sigma^{2}S_{t}^{2}\frac{\partial V}{\partial S}+\frac{\partial V}{\partial S}\mu S_{t}$$
\end{center}
\end{frame}

%------------------------------------------------
\begin{frame}
\frametitle{Last step in replication}
\begin{center}
$$\frac{\partial V}{\partial S}\mu S_{t} + rb_{t}M_{t} = \frac{\partial V}{\partial t}+\frac{1}{2}\sigma^{2}S_{t}^{2}\frac{\partial V}{\partial S}+\frac{\partial V}{\partial S}\mu S_{t}$$
$$rb_{t}M_{t} = \frac{\partial V}{\partial t}+\frac{1}{2}\sigma^{2}S_{t}^{2}\frac{\partial V}{\partial S}$$
$$ra_{t}S_{t}+rb_{t}M_{t} = \frac{\partial V}{\partial t}+\frac{1}{2}\sigma^{2}S_{t}^{2}\frac{\partial V}{\partial S}+ra_{t}S_{t}$$
$$rV_{t} = \frac{\partial V}{\partial t}+\frac{1}{2}\sigma^{2}S_{t}^{2}\frac{\partial V}{\partial S}+r\frac{\partial V}{\partial S}S_{t}$$
$$\frac{\partial V}{\partial t}+\frac{1}{2}\sigma^{2}S_{t}^{2}\frac{\partial V}{\partial S}+r\frac{\partial V}{\partial S}S_{t}- rV_{t} = 0$$
Hence,
$$\frac{\partial V}{\partial t}+\frac{1}{2}\sigma^{2}S^{2}\frac{\partial V}{\partial S}+r\frac{\partial V}{\partial S}S- rV = 0$$
\end{center}
\end{frame}

%-----------------------------------------------
\begin{frame}
\Huge{\centerline{Thank You}}
\begin{center}
\begin{normalsize}
\emph{E0007424@u.nus.edu}
\end{normalsize}
\end{center}
\end{frame}

%------------------------------------------------

\end{document} 