% This is a comment.
% the region directly below this comment, up till the command \begin{document} is known as the 'preamble'
% basic setup
\documentclass[12pt]{article}
\usepackage[english]{babel}
\usepackage[utf8]{inputenc}

% for mathematics
\usepackage{amsmath}
\usepackage{amsthm}
% define theorems, lemmas, etc
\newtheorem{theorem}{Theorem}
\newtheorem{lemma}{Lemma}
\newtheorem{corollary}{Corollary}
\newtheorem{definition}{Definition}
\newtheorem{example}{Example}
\usepackage{amssymb}

% for adjusting margins
\usepackage{geometry}
\geometry{
	a4paper,
 	left=26mm,
 	right=20mm,
 	top=33mm,
 	bottom=38mm
}

% for introducing urls
\usepackage{url}

% for colored text
\usepackage{color}

% for creating lists
\usepackage{enumerate}

% change font to times new roman
\usepackage{times}


%~~~~~~~~~~~~~~~~~~~~~~~~~~~~~~~~~~~~~~~~~~~~~~~~~~~~~~~~~~~~~~~~~~~~~~~~~~~~~~
% the region between \begin{document} ... \end{document} is known as the 'text'
\begin{document}
\begin{titlepage}
	\centering
	\vspace*{.09\textheight}
	{\LARGE\bfseries Undergraduate Research Opportunity\\
Programme in Science\par}
	\vspace{1.5cm}
	{\huge Financial Mathematics With Python\par}
	\vspace{1cm}
	{\scshape\Large Wang Zexin\par}
	\vspace{6cm}
	{\LARGE\scshape Zhou Chao\par}
	\vspace{3cm}
	{\Large Department of Mathematics\\[3mm]
National University of Singapore\\[3mm]
AY2016/17 Semester 2\par}
\end{titlepage}

\tableofcontents
\newpage

\section*{Abstract}
In this paper, we implemented the library packages for Python to solve problems in Financial Mathematics such as derivatives valuation and simulation as suggested by the book ``\emph{Python for Finance}''. Those results correspond to the different approaches in pricing financial derivatives. The key effort is on the development of valuation scheme for the options, in particular, the strongly path-dependent options. More importantly, our model is open to extend for various assumptions about the market. We attempt to apply these valuation schemes onto the path-independent options with available closed-form formulain order to verify the valuation results from Monte Carlo simulations as well as the Finite Difference Model.

\section{Introduction}
The study upon using Python to carry out derivatives valuation has made much progress over the years as there have already been available libraries built for this purpose. The derivatives analytics library suggested in the book ``\emph{Python for Finance}'' has its advantages on the coverage upon the various aspects of possible analysis for financial derivatives. Still, we can make justifiable modifications and extensions to improve on the accuracy and speed of computations for estimations. 
\\

\newpage

\section{Derivation of Black Scholes PDE}

\subsection{Basics}
\begin{center}
In the Black-Scholes World, we assume that the following two SDE hold:
$$dM_{t}  = rM_{t}dt$$
$$dS_{t}  = \mu S_{t}dt + \sigma S_{t} dW_{t}$$

We assume that the It$\hat{o}^{\prime}$s Lemma hold:\\[4mm]
As $dS_{t} = \mu S_{t}dt + \sigma S_{t}dW_{t}$,
$$dV_{t} = (\frac{\partial V}{\partial t}+\frac{1}{2}\sigma^{2}S_{t}^{2}\frac{\partial V}{\partial S})dt + \frac{\partial V}{\partial S}dS_{t}$$
\end{center}
\newpage

\subsection{Delta-hedging Argument}

\begin{center}
Our first aim is to find $\phi_{t}$\\[1mm]
for $\mathrm{\Pi}_{t} = V_{t} - \phi_{t}S_{t}$\\[3mm]
such that\\[1mm]
$d\mathrm{\Pi}_{t} = dV_{t} - \phi_{t}dS_{t}$(Self-financing)\\[1mm]
$d\mathrm{\Pi}_{t} = r\mathrm{\Pi}_{t}dt$(risk free)\\[3mm]
From the equations, we can obtain that\\[1mm]
$r\mathrm{\Pi}_{t}dt = dV_{t} - \phi_{t}dS_{t}$\\[1mm]
$r(V_{t} - \phi_{t}S_{t})dt = dV_{t} - \phi_{t}dS_{t}$\\[1mm]
$dV_{t}  = r(V_{t} - \phi_{t}S_{t})dt + \phi_{t}dS_{t}$\\[1mm]
By It$\hat{o}^{\prime}$s Lemma, $$dV_{t} = (\frac{\partial V}{\partial t}+\frac{1}{2}\sigma^{2}S_{t}^{2}\frac{\partial V}{\partial S})dt + \frac{\partial V}{\partial S}dS_{t}$$
Hence we obtain two equations:
$$\phi_{t} = \frac{\partial V}{\partial S}$$
$$r(V_{t} - \phi_{t}S_{t}) = \frac{\partial V}{\partial t}+\frac{1}{2}\sigma^{2}S_{t}^{2}\frac{\partial V}{\partial S}$$
\end{center}

\begin{center}
$$r(V_{t} - \frac{\partial V}{\partial S}S_{t}) = \frac{\partial V}{\partial t}+\frac{1}{2}\sigma^{2}S_{t}^{2}\frac{\partial V}{\partial S}$$
$$\frac{\partial V}{\partial t}+\frac{1}{2}\sigma^{2}S_{t}^{2}\frac{\partial V}{\partial S} + r\frac{\partial V}{\partial S}S_{t} - rV_{t} = 0$$
At time t, we have
$$\frac{\partial V}{\partial t}+\frac{1}{2}\sigma^{2}S^{2}\frac{\partial V}{\partial S} + r\frac{\partial V}{\partial S}S - rV = 0$$
\end{center}
\newpage

\subsection{Replicating portfolio}

\begin{center}
Our aim is to find $a_{t}$ and $b_{t}$ such that $\mathrm{\Pi}_{t} = a_{t}S_{t} + b_{t}M_{t}$ can entirely replicate $V_{t}$\\[2mm]
while the self-financing condition still holds: $d\mathrm{\Pi}_{t} = a_{t}dS_{t} + b_{t}dM_{t}$\\[5mm]
As $dS_{t} = \mu S_{t}dt + \sigma S_{t}dW_{t}$ and $dM_{t}  = rM_{t}dt$,
\begin{equation*}
\begin{split}
d\mathrm{\Pi}_{t} 
&= a_{t}(\mu S_{t}dt + \sigma S_{t}dW_{t}) + b_{t}(rM_{t}dt)\\
&= (a_{t}\mu S_{t} + rb_{t}M_{t})dt + (\sigma a_{t}S_{t})dW_{t}\\
\end{split}
\end{equation*}
\end{center}

\begin{center}
By It$\hat{o}^{\prime}$s Lemma, 
\begin{equation*}
\begin{split}
dV_{t}
&= (\frac{\partial V}{\partial t}+\frac{1}{2}\sigma^{2}S_{t}^{2}\frac{\partial V}{\partial S})dt + \frac{\partial V}{\partial S}dS_{t}\\
&= (\frac{\partial V}{\partial t}+\frac{1}{2}\sigma^{2}S_{t}^{2}\frac{\partial V}{\partial S})dt + \frac{\partial V}{\partial S}(\mu S_{t}dt + \sigma S_{t}dW_{t})\\
&= (\frac{\partial V}{\partial t}+\frac{1}{2}\sigma^{2}S_{t}^{2}\frac{\partial V}{\partial S}+\frac{\partial V}{\partial S}\mu S_{t})dt + (\sigma S_{t}\frac{\partial V}{\partial S})dW_{t}\\
\end{split}
\end{equation*}
\end{center}

\begin{center}
As $\mathrm{\Pi}_{t}$ fully replicates $V_{t}$,
$$d\mathrm{\Pi}_{t} = (a_{t}\mu S_{t} + rb_{t}M_{t})dt + (\sigma a_{t}S_{t})dW_{t} = dV_{t}$$
Also by It$\hat{o}^{\prime}$s Lemma, 
$$dV_{t} = (\frac{\partial V}{\partial t}+\frac{1}{2}\sigma^{2}S_{t}^{2}\frac{\partial V}{\partial S}+\frac{\partial V}{\partial S}\mu S_{t})dt + (\sigma S_{t}\frac{\partial V}{\partial S})dW_{t}$$
Hence we obtain that,
$$a_{t} = \frac{\partial V}{\partial S}$$
$$a_{t}\mu S_{t} + rb_{t}M_{t} = \frac{\partial V}{\partial t}+\frac{1}{2}\sigma^{2}S_{t}^{2}\frac{\partial V}{\partial S}+\frac{\partial V}{\partial S}\mu S_{t}$$
\end{center}

\begin{center}
$$\frac{\partial V}{\partial S}\mu S_{t} + rb_{t}M_{t} = \frac{\partial V}{\partial t}+\frac{1}{2}\sigma^{2}S_{t}^{2}\frac{\partial V}{\partial S}+\frac{\partial V}{\partial S}\mu S_{t}$$
$$rb_{t}M_{t} = \frac{\partial V}{\partial t}+\frac{1}{2}\sigma^{2}S_{t}^{2}\frac{\partial V}{\partial S}$$
$$ra_{t}S_{t}+rb_{t}M_{t} = \frac{\partial V}{\partial t}+\frac{1}{2}\sigma^{2}S_{t}^{2}\frac{\partial V}{\partial S}+ra_{t}S_{t}$$
$$rV_{t} = \frac{\partial V}{\partial t}+\frac{1}{2}\sigma^{2}S_{t}^{2}\frac{\partial V}{\partial S}+r\frac{\partial V}{\partial S}S_{t}$$
$$\frac{\partial V}{\partial t}+\frac{1}{2}\sigma^{2}S_{t}^{2}\frac{\partial V}{\partial S}+r\frac{\partial V}{\partial S}S_{t}- rV_{t} = 0$$
Hence,
$$\frac{\partial V}{\partial t}+\frac{1}{2}\sigma^{2}S^{2}\frac{\partial V}{\partial S}+r\frac{\partial V}{\partial S}S- rV = 0$$
\end{center}
\newpage
\subsection{Lognormal property}
We shall derive the lognormal property of the underlying asset prices.
As given by the Geometric Brownian Motion, $dS_{t}  = \mu S_{t}dt + \sigma S_{t} dW_{t}$.\\
We can define the derivative price function as $V(S_{t}, t) := \ln{S_{t}}$, with \\
$$\frac{\partial V}{\partial S} = \frac{1}{S}, \frac{\partial^{2} V}{\partial S^{2}} = -\frac{1}{S^{2}} , \frac{\partial V}{\partial t} = 0$$
By It$\hat{o}^{\prime}$s Lemma,
\begin{equation}
\begin{split}
dV_{t}
&= ((\mu S_{t})\frac{\partial V}{\partial S} + \frac{\partial V}{\partial t} + \frac{1}{2}\sigma^{2}S_{t}^{2}\frac{\partial^{2} V}{\partial S^{2}})dt + b\frac{\partial V}{\partial S}dW_{t}\\
&= (\mu - \frac{1}{2}\sigma^{2})dt + \sigma dW_{t}
\end{split}
\end{equation}
As $\mu$ and $\sigma$ are constants, $V_{t}$ is simply a drifted Brownian Motion.\\
$$V_{t} - V_{0} = \ln{S_{t}} - \ln{S_{0}} \sim \mathcal{N}((\mu - \frac{1}{2}\sigma^{2})t, \sigma^{2}t)$$
$$\frac{\ln{S_{t}} - \ln{S_{0}} - (\mu - \frac{1}{2}\sigma^{2})t}{\sigma\sqrt{t}} \sim \mathcal{N}(0,1)$$
We can also generalize to the case for time $t$ and $T$. (Let $\tau = T - t$, $\phi \sim \mathcal{N}(0,1)$)
$$\frac{\ln{S_{T}} - \ln{S_{t}} - (\mu - \frac{1}{2}\sigma^{2})\tau}{\sigma\sqrt{\tau}} = \phi $$
$$S_{T} = S_{t}e^{(\mu - \frac{1}{2}\sigma^{2})\tau + \sigma\sqrt{\tau}\phi}$$
\newpage

\section{Finite Difference Model for Numerical PDE}
$$\frac{\partial V}{\partial t}+\frac{1}{2}\sigma^{2}S^{2}\frac{\partial V}{\partial S}+r\frac{\partial V}{\partial S}S- rV = 0$$
If we are to plot out the option prices in a diagram, there should be a 3D diagram with y-axis being the underlying prices, and x-axis being the time, omitting the scenario when the option price is dependent on other stochastic terms such as volatility.
In the Black-Scholes formula, we are able to approximate $\Theta = \frac{\partial V}{\partial t}$, $\Delta = \frac{\partial V}{\partial S}$ and $\Gamma = \frac{\partial^{2} V}{\partial S^{2}}$ when the discretization is significantly small and thus it will be easy to simply use the ratio of change in derivatives value with respect to change in time or change in underlying asset value.\\
$$\frac{\partial V}{\partial t} \approx \frac{P_{t} - P_{t-{^{\Delta}t}}}{^{\Delta}t}$$
$$\frac{\partial V}{\partial S} \approx \frac{P_{S} - P_{S-{^{\Delta}S}}}{^{\Delta}S} \approx \frac{P_{S+{^{\Delta}S}} - P_{S}}{^{\Delta}S}$$
\begin{equation}
\begin{split}
\frac{\partial^{2} V}{\partial S^{2}}
&\approx \frac{\partial}{\partial V}(\frac{P_{S+{^{\Delta}S}} - P_{S}}{^{\Delta}S})\\
&\approx \frac{\frac{P_{S+{^{\Delta}S}} - P_{S}}{^{\Delta}S} - \frac{P_{S} - P_{S-{^{\Delta}S}}}{^{\Delta}S}}{^{\Delta}S}\\
&= \frac{P_{S+{^{\Delta}S}} - 2P_{S} + P_{S-{^{\Delta}S}}}{(^{\Delta}S)^{2}}
\end{split}
\end{equation}

\subsection{One factor model using Implicit Scheme}
\newpage
\subsection{Explicit Scheme}
The difference between explicit and implicit scheme lies in the need to solve an implicit equation at each discretization or there is a closed-form formula for each step.
\newpage
\subsection{Feynman–Kac formula}
\newpage
\subsection{Crank–Nicolson method}
\newpage

\section{Variance Reduction Techniques for Monte Carlo Simulation}

\subsection{Control Variate}
\newpage
\subsection{Stratified Sampling}
\newpage
\subsection{Importance Sampling}
\newpage

\section{European call options}
\subsection{Closed-form formula}
We can derive the closed-form formula for the European call options using risk-neutral pricing.\\
Payoff function of the European call option is : $c_{T} = (S_{T} - K)^{+}$\\
By the risk-neutral pricing formula,
\begin{equation}
\begin{split}
e^{-\mu t}c_{t}
&= \mathbb{E}^{Q}[e^{-\mu T}c_{T} | \mathcal{F}_{t}]\\
&= \mathbb{E}_{t}^{Q}[e^{-\mu T}(S_{T} - K)^{+}]
\end{split}
\end{equation}
By the lognormal property, \quad
$S_{T} = S_{t}e^{(\mu - \frac{1}{2}\sigma^{2})\tau + \sigma\sqrt{\tau}\phi}$
\begin{equation}
\begin{split}
c_{t} 
&= e^{\mu t}\mathbb{E}_{t}^{Q}[e^{-rT}(S_{T} - K)^{+}]\\
&= e^{\mu t}\mathbb{E}_{t}^{Q}[e^{-rT}(S_{t}e^{(\mu - \frac{1}{2}\sigma^{2})\tau + \sigma\sqrt{\tau}\phi} - K)^{+}]\\
&= e^{-\mu \tau}\mathbb{E}^{Q}[(S_{t}e^{(\mu - \frac{1}{2}\sigma^{2})\tau + \sigma\sqrt{\tau}\phi} - K)^{+}]\\
&= e^{-\mu \tau}\int_{-\infty}^{\infty} (S_{t}e^{(\mu - \frac{1}{2}\sigma^{2})\tau + \sigma\sqrt{\tau}\phi} - K)^{+}\frac{1}{\sqrt{2\pi}}e^{-\frac{1}{2}\phi^{2}}  \, d\phi \\
&= e^{-\mu \tau}\int_{-d_{2}}^{\infty} (S_{t}e^{(\mu - \frac{1}{2}\sigma^{2})\tau + \sigma\sqrt{\tau}\phi} - K)\frac{1}{\sqrt{2\pi}}e^{-\frac{1}{2}\phi^{2}}  \, d\phi \\
&= e^{-\mu \tau}\int_{-d_{2}}^{\infty} S_{t}e^{(\mu - \frac{1}{2}\sigma^{2})\tau + \sigma\sqrt{\tau}\phi}\frac{1}{\sqrt{2\pi}}e^{-\frac{1}{2}\phi^{2}}  \, d\phi - e^{-r\tau}\int_{-d_{2}}^{\infty} \frac{1}{\sqrt{2\pi}}e^{-\frac{1}{2}\phi^{2}}K\, d\phi\\
&= e^{(\mu - \frac{1}{2}\sigma^{2})\tau-\mu \tau}S_{t}\int_{-d_{2}}^{\infty} \frac{1}{\sqrt{2\pi}}e^{-\frac{1}{2}(\phi^{2}-2\sigma\sqrt{\tau}\phi)}  \, d\phi - Ke^{-r\tau}\int_{-d_{2}}^{\infty} \frac{1}{\sqrt{2\pi}}e^{-\frac{1}{2}\phi^{2}}\, d\phi\\
&= S_{t}\int_{-d_{2}}^{\infty} \frac{1}{\sqrt{2\pi}}e^{-\frac{1}{2}(\phi^{2}-2\sigma\sqrt{\tau}\phi+\sigma^{2}\tau)}  \, d\phi - Ke^{-r\tau}\int_{-\infty}^{d_{2}} \frac{1}{\sqrt{2\pi}}e^{-\frac{1}{2}\phi^{2}}\, d\phi\\
&= S_{t}\int_{-d_{2}}^{\infty} \frac{1}{\sqrt{2\pi}}e^{-\frac{1}{2}(\phi-\sigma\sqrt{\tau}\phi)^{2}}  \, d\phi - Ke^{-r\tau}\int_{-\infty}^{d_{2}} \frac{1}{\sqrt{2\pi}}e^{-\frac{1}{2}\phi^{2}}\, d\phi\\
&= S_{t}\int_{-d_{2}-\sigma\sqrt{\tau}}^{\infty} \frac{1}{\sqrt{2\pi}}e^{-\frac{1}{2}y^{2}}  \, dy - Ke^{-r\tau}\int_{-\infty}^{d_{2}} \frac{1}{\sqrt{2\pi}}e^{-\frac{1}{2}\phi^{2}}\, d\phi\\
&= S_{t}\int_{-\infty}^{d_{2}+\sigma\sqrt{\tau}} \frac{1}{\sqrt{2\pi}}e^{-\frac{1}{2}y^{2}}  \, dy - Ke^{-r\tau}\int_{-\infty}^{d_{2}} \frac{1}{\sqrt{2\pi}}e^{-\frac{1}{2}\phi^{2}}\, d\phi\\
&= S_{t} N(d_{1}) - Ke^{-r\tau} N(d_{2})
\end{split}
\end{equation}
where,
$$d_{2} = \frac{\ln{\frac{S_{t}}{K}} + (\mu - \frac{1}{2}\sigma^{2})\tau}{\sigma\sqrt{\tau}}$$
$$d_{1} = d_{2} + \sigma\sqrt{\tau}$$
Also, $N$ is the cumulative density function of standard normal function.
\newpage
\subsection{Numerical PDE}
\newpage
\subsection{Monte Carlo Simulation}
\newpage

\section{European put options}

\subsection{Closed-form formula}
We can derive the closed-form formula for the European put options using risk-neutral pricing.\\
Payoff function of the European put option is : $p_{T} = (K - S_{T})^{+}$\\
By the risk-neutral pricing formula,
\begin{equation}
\begin{split}
e^{-\mu t}p_{t}
&= \mathbb{E}^{Q}[e^{-\mu T}p_{T} | \mathcal{F}_{t}]\\
&= \mathbb{E}_{t}^{Q}[e^{-\mu T}(K - S_{T})^{+}]
\end{split}
\end{equation}
By the lognormal property, \quad
$S_{T} = S_{t}e^{(\mu - \frac{1}{2}\sigma^{2})\tau + \sigma\sqrt{\tau}\phi}$
\begin{equation}
\begin{split}
p_{t} 
&= e^{\mu t}\mathbb{E}_{t}^{Q}[e^{-rT}(K - S_{T})^{+}]\\
&= e^{\mu t}\mathbb{E}_{t}^{Q}[e^{-rT}(K - S_{t}e^{(\mu - \frac{1}{2}\sigma^{2})\tau + \sigma\sqrt{\tau}\phi})^{+}]\\
&= e^{-\mu \tau}\mathbb{E}^{Q}[(K - S_{t}e^{(\mu - \frac{1}{2}\sigma^{2})\tau + \sigma\sqrt{\tau}\phi})^{+}]\\
&= e^{-\mu \tau}\int_{-\infty}^{\infty} (K - S_{t}e^{(\mu - \frac{1}{2}\sigma^{2})\tau + \sigma\sqrt{\tau}\phi})^{+}\frac{1}{\sqrt{2\pi}}e^{-\frac{1}{2}\phi^{2}}  \, d\phi \\
&= e^{-\mu \tau}\int_{-\infty}^{-d_{2}} (K - S_{t}e^{(\mu - \frac{1}{2}\sigma^{2})\tau + \sigma\sqrt{\tau}\phi})\frac{1}{\sqrt{2\pi}}e^{-\frac{1}{2}\phi^{2}}  \, d\phi \\
&= e^{-r\tau}\int_{-\infty}^{-d_{2}} \frac{1}{\sqrt{2\pi}}e^{-\frac{1}{2}\phi^{2}}K\, d\phi -e^{-\mu \tau}\int_{-\infty}^{-d_{2}} S_{t}e^{(\mu - \frac{1}{2}\sigma^{2})\tau + \sigma\sqrt{\tau}\phi}\frac{1}{\sqrt{2\pi}}e^{-\frac{1}{2}\phi^{2}}  \, d\phi\\
&= Ke^{-r\tau}\int_{-\infty}^{-d_{2}} \frac{1}{\sqrt{2\pi}}e^{-\frac{1}{2}\phi^{2}}\, d\phi - e^{(\mu - \frac{1}{2}\sigma^{2})\tau-\mu \tau}S_{t}\int_{-\infty}^{-d_{2}} \frac{1}{\sqrt{2\pi}}e^{-\frac{1}{2}(\phi^{2}-2\sigma\sqrt{\tau}\phi)}  \, d\phi\\
&= Ke^{-r\tau}\int_{-\infty}^{-d_{2}} \frac{1}{\sqrt{2\pi}}e^{-\frac{1}{2}\phi^{2}}\, d\phi - S_{t}\int_{-\infty}^{-d_{2}} \frac{1}{\sqrt{2\pi}}e^{-\frac{1}{2}(\phi^{2}-2\sigma\sqrt{\tau}\phi+\sigma^{2}\tau)}  \, d\phi\\
&= Ke^{-r\tau}\int_{-\infty}^{-d_{2}} \frac{1}{\sqrt{2\pi}}e^{-\frac{1}{2}\phi^{2}}\, d\phi - S_{t}\int_{-\infty}^{-d_{2}} \frac{1}{\sqrt{2\pi}}e^{-\frac{1}{2}(\phi-\sigma\sqrt{\tau}\phi)^{2}}  \, d\phi\\
&= Ke^{-r\tau}\int_{-\infty}^{d_{2}} \frac{1}{\sqrt{2\pi}}e^{-\frac{1}{2}\phi^{2}}\, d\phi - S_{t}\int_{-\infty}^{-d_{2}-\sigma\sqrt{\tau}} \frac{1}{\sqrt{2\pi}}e^{-\frac{1}{2}y^{2}}  \, dy\\
&= Ke^{-r\tau}\int_{-\infty}^{d_{2}} \frac{1}{\sqrt{2\pi}}e^{-\frac{1}{2}\phi^{2}}\, d\phi - S_{t}\int_{-\infty}^{-d_{1}} \frac{1}{\sqrt{2\pi}}e^{-\frac{1}{2}y^{2}}  \, dy \\
&= Ke^{-r\tau} N(-d_{2}) - S_{t} N(-d_{1})
\end{split}
\end{equation}
where,
$$d_{2} = \frac{\ln{\frac{S_{t}}{K}} + (\mu - \frac{1}{2}\sigma^{2})\tau}{\sigma\sqrt{\tau}}$$
$$d_{1} = d_{2} + \sigma\sqrt{\tau}$$
Also, $N$ is the cumulative density function of standard normal function.
\newpage
\subsection{Numerical PDE}
\newpage
\subsection{Monte Carlo Simulation}
\newpage

\section{Barrier option}

\subsection{Closed-form formula}
\newpage
\subsection{Numerical PDE}
\newpage
\subsection{Monte Carlo Simulation}
\newpage

\end{document}